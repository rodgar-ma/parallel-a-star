\documentclass{article}
\usepackage[utf8]{inputenc}

\title{Implementación Física de un Procesador Cuántico. Actividad 2.}
\author{Miguel Ángel Rodríguez García\\José Izquierdo González}
\date{}

\begin{document}
\maketitle

\section{Conclusiones de la actividad.}

En la actividad desarrollada se han debatido a través del foro de la palataforma de UNIR las distintas tecnologías propuestas a día de hoy como candidatos a una implementación eficiente y escalable de cúbits. En nuestra aportación defendíamos las ventajas de los cúbits superconductores, y tras revisar las intervenciones de nuestros compañeros, hemos llegado a las siguientes conclusiones.

La exploración de las distintas tecnologías de cúbits pone de manifiesto que aún no existe una solución única y definitiva para la construcción de un ordenador cuántico práctico. Cada enfoque presenta fortalezas particulares, lo que ha generado una diversidad de opiniones dentro del foro. Algunos compañeros han defendido con acierto el potencial de tecnologías como los átomos neutros o las trampas de iones, destacando aspectos como la alta fidelidad, los largos tiempos de coherencia o la capacidad de escalado mediante arquitecturas modulares.

Por ejemplo, en el caso de los átomos neutros, se ha subrayado su gran proyección en cuanto a escalabilidad, así como su aplicabilidad en simulaciones cuánticas, gracias a la posibilidad de organizar grandes redes de cúbits mediante pinzas ópticas. Sin embargo, tal y como también han reconocido quienes defienden esta tecnología, aún existen limitaciones importantes en cuanto a fidelidad de puertas y velocidad de operación, lo que dificulta su uso actual en escenarios digitales exigentes.

Del mismo modo, los compañeros que han abogado por las trampas de iones han puesto en valor la altísima precisión de las operaciones y la conectividad total entre cúbits, lo que sin duda representa una ventaja significativa. No obstante, también han apuntado que la lentitud de las puertas cuánticas y los retos técnicos asociados al escalado del sistema siguen siendo obstáculos relevantes para su implementación a gran escala.

Frente a estos enfoques, nosotros consideramos que los cúbits superconductores representan, en el estado actual de la tecnología, la opción más sólida y realista a medio plazo. Aunque requieren operar a temperaturas extremadamente bajas y presentan desafíos relacionados con la coherencia, su desarrollo se encuentra claramente más avanzado. Las puertas cuánticas en estas plataformas han alcanzado velocidades muy superiores a las de otras tecnologías, y en muchos casos se ha logrado superar el 99\% de fidelidad en operaciones de dos cúbits. Además, cuentan con el respaldo de una infraestructura tecnológica bien consolidada, impulsada por empresas líderes como IBM y Google, lo que ha permitido avances significativos en ámbitos como la corrección de errores cuánticos y la integración de sistemas a mayor escala.

En definitiva, si bien reconocemos el valor de los argumentos presentados por otros compañeros y coincidimos en que el futuro de la computación cuántica probablemente será híbrido —combinando distintas tecnologías según el tipo de problema o aplicación—, creemos que los cúbits superconductores ofrecen actualmente la mejor combinación de velocidad, fidelidad, escalabilidad y respaldo tecnológico. Por todo ello, pensamos que seguirán siendo la base principal para el desarrollo de los primeros ordenadores cuánticos funcionales y útiles en los próximos años.
\end{document}